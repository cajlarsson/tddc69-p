\documentclass[a4paper,8pt]{article}
\usepackage[utf8]{inputenc}
\usepackage[T1]{fontenc}      
\usepackage[swedish]{babel}
\usepackage{inconsolata}

\begin{document}

\begin{titlepage}

TDDC69 - Objektorienterad programmering och Java
\vspace {4cm}

\begin{center}

 
\Huge{\bfseries Projektspecifikation}
\vspace {1cm}

\LARGE{\bfseries ``Cave Explorer''}
\vspace {1cm}

\LARGE{\bfseries 2010-09-20}
\vspace {1cm}

\large{ \bfseries Projektmedlemmar:}

\large{ Caj Larsson cajla304@student.liu.se}

\large{ Erik Lindholm erili322@student.liu.se}
\vspace {1cm}

\large{ \bfseries Handledare:}

\large{ Handledare handledare@ida.liu.se}
\end{center}
\end{titlepage}

%innehåållsförteckning
\tableofcontents
\pagebreak

\section{Inledning}
%Den här delen skriver ni i samband med första inlämningen
%Ge en kort inledning till projektet (vad är det för spel, vad går det ut på, …).
%Det ska vara kort och koncist men ge en tillräcklig förklaring till vad spelet 
%går ut på.
Cave explorer ska bli ett spel där två lag kämpar mot varandra under jorden över
kontrollen över kartan. Det ska vara ett 2D 'RTS' där abstraherad tid ska vara
den ända valutan, all tid du inte spenderar ackumuleras.

\subsection{Spelbeskrivning}
%Ge en lite mer genomgående förklaring av spelet och hur det spelas. Till 
%exempel: "Brickor trillar ned från över delen av skärmen. Dessa kan roteras och
%målet är att stapla dessa i nedre delen av skärmen och bilda rader.". Eventuella
% regler för spelet är även bra att förklara här (för schack till exempel).
Vardera spelare börjar med varsin underjodsbas där de kan skapa soldater som
antingen kan gräva eller strida med en varierad arsenal. För att vinna måste
dina soldater förstöra motståndarens bas. För att nå motsåndarens bas så
måste tunnlar grävas mellan baserna så att dina soldater kan nå fram.
Tunnlar kollapsar efter en viss tid om de inte underhålls. Soldater kan bara
skjuta i fyra riktningar och kan bara styras med köade ordrar, vi har även
planer på att införa order-macron.

\subsection{Utvecklingsmetodik}
%Hur tänker ni jobba? Ska ni dela upp hur ni skriver koden? Hur kommunicerar ni?
%Hur delar ni koden mellan varandra och ser till att den är tillgänglig, även om 
%en projektmedlem är sjuk eller borta (ett vanligt problem som uppstår, därför är
%det viktigt att ni planerar inför detta tidigt)? Tänker ni versionshantera koden
%(till exempel med subversion eller git)? Jobba kvällar? Helger? Hur lång tid 
%tror ni projektet kommer kräva? Vad har ni för betygsambitioner?
Vi kommer att dela upp koden i olika delprojekt ala grafikmotor spelmotor och
grafik, sen kommer vi att dela upp vem som har huvudansvar för varje delprojekt.
Antagligen kommer vi byta kod och uppgifter med varandra under utvecklandet. Vi
kommer antagligen inte ha några problem med frånvaro osv, vi brukar lösa det 
utan problem. Vi kör git på hela projektet, inklusive denna dokumentation. Det
kommer antagligen ta mer tid än utsatt i kursen innan vi har kommit så långt 
att vi är nöjda. Vi har femma som mål.


%\section{Användarmanual}
%När ni har implementerat ett spel så krävs det också en manual som förklarar hur
%spelet fungerar. Ni ska beskriva spelets olika delar och hur interaktionen 
%fungerar (om mus används, tangentbord, vilka knappar som är aktuella, etc). 
%Inkludera skärmdumpar, som visar hur spelet ser ut. Det ska åtminstone finnas en
%icke-modifierad bild av spelet. Ett tips i övrigt kan vara att ha en extra bild,
%ringa in och markera intressanta områden med siffror/färger och beskriva dessa i
%en prydlig tabell.

\section{Kravlista}
%Lista de krav som ni har på spelet. Dessa ska vara mätbara, det vill säga att 
%man ska kunna svara "ja" eller "nej" på om kravet är uppfyllt eller ej. Ett 
%exempel på ett bra krav är: "Spelet skall styras med tangentbordet", eftersom
%det är väldigt lätt att svara ja eller nej på det. Ett dåligt krav är: "Spelet
%skall vara snyggt", för om spelet är snyggt eller ej är en högst subjektiv 
%bedömning. Varje krav ska ha ett unikt nummer, som börjar räknas från 1 och
%uppåt. Dela också upp kraven i "skall"- och "bör"-krav (genom att ge kraven
%högre eller lägre prioritering). Skall-kraven är de som ni måste implementera
%för att projektet skall anses vara lyckat (ni får förmodligen inte underkänt om
%inte uppfyller dessa; ni jobbar mest principiellt här). Bör-kraven är sådana ni
%implementerar om ni får tid över. Ett smidigt sätt att organisera alla krav är
%genom att bygga ut tabellen nedan.
%
%\begin{tabular}{|l |l |l|}
%  \hline
%  \# & Beskrivning & Prioritet \\ 
%  \hline
%  1 & Spelaren skall styra med musen & 1 \\
%  \hline
%\end{tabular}
%
%Även om det finns många triviala ”krav” som kan läggas till (”spelet skall 
%styras med musen”, ...), försök hitta bra krav som återspeglar de 
%karaktäristiska dragen som finns hos ert spel. Eftersom ni under kapitlet 
%”Arbetsmetodik” har redogjort för önskat betyg, så kan ni genom kraven skapa en 
%övertygelse om att det är just det betyget ni ska ha. Vid osäkerhet kan ni fråga
%er handledare om råd och fist

Våra krav:

\vspace{0.2cm}

\begin{tabular}{|l |l |l|}
  \hline
 \bfseries \# &\bfseries Beskrivning &\bfseries Prioritet \\ 
  \hline
  1 & Simpel spelvärld med grafik & 1 \\  
  \hline
  2 & Kollisionshantering och enheter & 1 \\
  \hline
  3 & Spelvärlds modifikation & 1 \\
  \hline
  4 & Skapning och avskapning av enheter & 1 \\
  \hline
  5 & Ordergivning(en order och eval) & 1 \\
  \hline
  6 & Scorescreen & 1 \\
  \hline
  7 & Modulärt spelar/adminfönster & 1 \\
  \hline
  8 & Kösys för ordrar & 2 \\
  \hline
  9 & Simpel AI & 2 \\
  \hline
  10 & Varierad Arsenal & 2 \\
  \hline
  11 & 'Byggnader' & 2 \\
  \hline
  12 & Raserbara gångar & 2 \\
  \hline
  13 & Underhåll och degeneration av gångar & 2 \\
  \hline
  14 & Meny & 3 \\
  \hline
  15 & Inställbar kartstorlek & 3 \\
  \hline
  16 & Nätverk & 3 \\
  \hline
  17 & Inspelning/reprissystem & 4 \\
  \hline
  18 & Flera plan & 5 \\
  \hline
  19 & FeatureSpam & 6 \\

  \hline
\end{tabular}

%\section{Implementation}
%Det här kapitlet använder ni för att beskriva hur spelet är strukturerat och
%implementerat. Dels ska ni förklara de klasser ni använder, men även hur dessa 
%används i er kod. Algoritmer och övergripande design passar också in i det här
%kapitlet (bilder, flödesdiagram, osv. är rekommenderat men ej något krav). Den 
%här delen kan ni strukturera upp enligt egna preferenser. Skapa gärna egna 
%delkapitel för enskilda delar, om det underlättar.

\section{Klasser}
Vi behöver ett spelvärlds objekt som innehåller flera rutobjekt som kan vara
både jordtäckta eller utgrävda och kan ifall de är utgrävda ha andra objekt
som enheter o dyl på sig. Enheter kan ha olika objekt som är vapen tex spade 
men bara ett i taget. Det behövs en spelserver som har koll på tiden och
skickar tick till alla tidsberoende objekt.

Spelarklassen har ett basstationsobjekt och enhetsobjekt, den ska vara 
abstrakt och både AI:n och spelarensgränsnitt skall uttnyttja den.

Enheter och allt annat som skall kunna klickas på måste implementera ett 
\texttt{interface} som anger vad den skall göra med de paneler som skall 
ändras när den är markerad.

Speletsflöden kommer kontrolleras av spelmotorn som uppdaterar alla 
spelareobjekt som i sin tur uppdaterar sitt fönster om det är applicerbart.
Spelarklasserna skickar förfrågningar om ordrar till spelmotorn som antingen
avvisar dem eller godkänner och köar dem.

Spelarfönstret kommer ha 3 olika signifikanta ytor, en panel för spelarens och
dess enheters status, tunn och ovanför allt annat. En sido panel med t ex 
enhets information. Samt en stor spelyta där spelet ritas ut.


%Den här delen skriver ni i samband med första inlämningen
%
%Fundera ut vilka klasser ni behöver för att implementera spelet. Beskriv 
%ytterligare hur klasserna används i er implemenation (väldigt intressant för
%klasser som hör till något designmönter). Använd klassdiagram (UML) för att 
%beskriva klasserna och dess relationer till varandra. Det kan vara bra att 
%försöka dela upp klassdiagrammet i flera mindre delar. Det finns annars en 
%överhängande risk att diagrammen växer i storlek snabbt och blir svåra att 
%överblicka, vilket gör att klassdiagrammen inte gör så mycket nytta. Detta 
%kapitel behöver inte vara komplett nu, det viktigaste är att ni börjat tänka på
%hur ni ska göra. Om ni jobbar mycket med detta kapitel så är det i stort sett 
%bara att sätta igång och koda sen, vilket gör att ni sparar mycket tid. Denna 
%delen fyller ni på ytterligare när ni lämnar in dokumentet igen i slutet av kursen.

%\subsection{ Beskrivning av implementationen}
%Den här delen skriver ni inför slutinlämningen
%
%Här ska ni förklara hur era datastrukturer och algoritmer fungerar. Det bästa 
%sättet att tänka är lite grann ur någon annan grupps perspektiv. Vad skulle 
%någon annan behöva veta för att förstå en viss del av spelet eller dess 
%implementation? Ni får själva försöka hitta alla dessa delar genom att ställa er
%själva frågor och studera er kod. Till exempel: "Vad behöver jag veta för att 
%förstå hur rotationsproceduren i mitt Tetris fungerar?".
%
%Tanken är inte att ni ska skriva och svara på frågor. Detta kan ni snarare se 
%som hjälp för att skriva denna delen.
%
%\section{Utvärdering och erfarenheter}
%Den här delen skriver ni inför slutinlämningen
%
%Detta avsnittet är en väldigt viktig del av projektspecifikationen. Här ska ni 
%tänka tillbaka och utvärdera projektet (något som alltid ska göra efter ett 
%projekt). Som en hjälp på vägen kan ni utgå från följande frågeställningar:
%\begin{itemize}
%\item Vad gick bra? Mindre bra?
%\item Lade ni ned för mycket/lite tid? 
%\item Var arbetsfördelningen jämn? Om inte: Vad hade ni kunnat göra för att 
%förbättra den?
%\item Har ni haft någon nytta av specifikationen? Vad har varit mest användbart
%med den? Minst?
%\item Har arbetet fungerat som ni tänkt er? Har ni följt "arbetsmetodiken"? 
%Något som skiljer sig? Till det bättre? Till det sämre?
%\item Vad har varit mest problematiskt, om man utesluter den 
%programmeringstekniska delen? Alltså saker runt omkring, som att hitta ledig tid
%eller plats att vara på.
%\item Vad har ni lärt er så här långt som kan vara bra att ta med till kommande kurser/projekt?
%\end{itemize}
\end{document}
